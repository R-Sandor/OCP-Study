%file: chapter1.tex
\documentclass[12pt]{article}
\usepackage{blindtext}
\usepackage[T1]{fontenc}
\usepackage{listings} %code extracts
\usepackage{xcolor} %custom colors
\usepackage{mdframed} %nice frames
\usepackage{lmodern}
\usepackage{tikz}
\usepackage{tabularx, ragged2e, booktabs}
\usepackage{makecell}
\newcolumntype{Z}{ >{\centering\arraybackslash}X }
\definecolor{light-gray}{gray}{0.95} %the shade of grey that stack exchange uses

\def\checkmark{\tikz\fill[scale=0.4](0,.35) -- (.25,0) -- (1,.7) -- (.25,.15) -- cycle;}
\begin{document}
\title {Chapter 1 OCP Notes}
\author {Raphael J. Sandor}
\maketitle
\subsection*{Java Class Design}
\begin{itemize} 	
	\item Implement inheritance including visibility modifier and composition. \checkmark
	\item Implement polymorphism. \checkmark
	\item Override hashCode, equals, and toString methods from Object class. \checkmark
	\item Develop code that uses the static keyword on initialize blocks, variables, and classes. \checkmark
\end{itemize}

\subsection*{Advanced Java Class Design }
\begin{itemize} 	
	\item Develop code that uses abstract classes and methods.  \checkmark
	\item Develop code that uses final keyword. \checkmark
	\item Create inner classes including static inner class, local class, nested class, and anonymous inner class.  \checkmark
	\item Use enumerated types including methods and constructors in an enum type. \checkmark
	\item Develop code that declares, implements, and/or extends interface and uses the @Override annotation. \checkmark
\end{itemize}

\section*{Notes}

\section{Access Modifiers}

\begin{table}[h]																				% placement parameter.
\setlength\extrarowheight{2pt} 
\caption{Access Modifiers}	% title of Table
\centering																						% used for centering table.
\begin{tabularx}{\textwidth}{*{5}{Z}}
\toprule
%\multicolumn{5}{c}{xxx xx xxxx}  \\
\thead{Can access}  
& \thead{Member \\ is private} 
& \thead{Member \\ has Default \\ (package \\ private)\\ access} 
& \thead{Member \\ is protected} 
& \thead{Member \\ is public}  \\ 
%heading
\hline																							% inserts single horizontal line
Member in the same class 						& Yes	& Yes	& Yes	& Yes		\\			% inserting body of the table 
\midrule
Member in another class in the same package		& No	& Yes	& Yes	& Yes		\\ [1ex]	% [1ex] adds vertical space
\midrule
Member in a superclass in a different package	& No 	& No	& Yes	& Yes		\\
\midrule
Method/field in a class (that is not a superclass) in a different package
												& No	& No	& No	& Yes		\\
\end{tabularx}	
\label{table:nonlin}																			
\end{table}

\section{Overloading and Overriding}

\subsection{Overloading}
When multiple overloaded methods are present, Java looks for the closet match first. It tries to find the following:
\begin{itemize}
  \item Exact match by type
  \item Matching a superclass
  \item Converting to larger primitive type
  \item Converting to an autoboxed type
  \item Varargs
\end{itemize}

\subsection{Overriding}
For overriding, the overriden method has few rules:

\begin{itemize}
  \item The access modifier must be the same or more accessible
  \item The return type must be the same same or more restrictive type, \\
   also known as \textit{covariant} return types.
  \item If any checked exceptions are thrown, only the same exceptions or subclass of those \\
   exceptions are allowed to be thrown.
 \end{itemize}


\subsection{OCA Concepts Notes} 
\begin{itemize}
  \item instanceOf will not compile for Objects that share no inheritance, exception this is of course interfaces and null.
    \subitem Reason for this is any object could be set null, and any class could at some point implement a given interface.
  \item equals() - Returns \textbf{boolean}, takes a single \textbf{Object obj} as a parameter.
  \item hashCode() - Returns \textbf{int}, takes \textbf{no parameters}.
  \item toString() - Returns a \textbf{String} object, takes \textbf{no parameters}.
\end{itemize}

\subsection* {Order of initialization} 
\begin{itemize}
  \item Superclass
  \item Static variable delcarations and static initializers in the order they appear in the file.
  \item Instance variable declarations and instance initializers in the order they appear in the file.
  \item The constructor
\end{itemize}



\end{document}
