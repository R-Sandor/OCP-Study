%file: chapter4.tex
\documentclass[12pt]{article}
\usepackage{blindtext}
\usepackage[T1]{fontenc}
\usepackage{listings} %code extracts
\usepackage{xcolor} %custom colors
\usepackage{mdframed} %nice frames
\usepackage{lmodern}
\usepackage{tikz}

\definecolor{light-gray}{gray}{0.95} %the shade of grey that stack exchange uses

\def\checkmark{\tikz\fill[scale=0.4](0,.35) -- (.25,0) -- (1,.7) -- (.25,.15) -- cycle;}
\begin{document}
\title {Chapter 4 OCP Notes}
\author {Raphael J. Sandor}
\maketitle
\subsection*{Generics and collections:}
\begin{itemize} 	
	\item Collection Streams and Filters. %\checkmark
	\item Iterate using forEach methods of Streams and List. %\checkmark
	\item Describe Stream interface and Stream pipeline. %\checkmark
	\item Use method references. \checkmark
\end{itemize}

\subsection*{Lambda Built-in Functional Interfaces:}
\begin{itemize} 	
	\item Use built-in interfaces included in the java.util.functional package
		  such as Predicate, Consumer, Function, and Supplier. \checkmark
	\item Develop code that uses primitive versions of functional interfaces. \checkmark
	\item Develop code that uses binary versions of functional interfaces. \checkmark
	\item Develop code that uses the UnaryOperator interface. \checkmark
\end{itemize}

\subsection*{Java Stream API:}
\begin{itemize} 	
		\item Develop code to extract data from an object using peek();
			and map() methods including primitive versions of the map() method. %\checkmark
		\item Search for data by using search methods of the Stream classes including 
			  findFirst, findAny, anyMatch, allMatch, and noneMatch.%\checkmark
		\item Develop code that uses Stream data methods and calculation methods.
		\item Sort a collection using Stream API. %\checkmark
		\item Save results to a collection using the collect method and group/partition/data using the Collectors class. %\checkmark
		\item Use of merge() and flatMap() methods of the Stream API. %\checkmark
\end{itemize}

\
\section*{Notes}


\section{Effectively Final}
If a local variable can have final put in the declaration and still compile, it is said to be effectively final.
\subsection{Using variables in Lambdas } 
Lambda expressions can access 
\begin{itemize}
	\item Static variables
	\item Instance variables
	\item Effectively final method parameters
	\item Effectively final local parameters
\end{itemize}
Lambdas can not access private variables in another class. 
\\Lambdas can access a subset of variables that are accessible, but never more than that.
\clearpage

\section{Working with Built in Function Interfaces } 
java.util.function.package
\\

\begin{table}[ht]
\caption{Common Functional Interfaces}	% title of Table
%\centering											% used for centering table
\begin{tabular}{llll}							% centered columns (4 columns)
\hline\hline										%inserts double horizontal lines
Functional Interfaces & \#Parameters & Return type & Single Abstract Method \\ [.5ex]
%heading
\hline																% inserts single horizontal line
Supplier<T>			& 0			& T			& get	  	\\			% inserting body of the table 
Consumer<T>			& 1 (T) 	& void		& accept  	\\ [1ex]	% [1ex] adds vertical space
BiConsumer<T>		& 2 (T, U) 	& void		& accept	\\
Predicate<T>		& 1 (T)		& boolean	& test		\\
BiPredicate<T, U>	& 2 (T, U)	& boolean	& test		\\
Function<T, R>		& 1 (T)		& R			& apply		\\
BiFunction<T, U, R> & 2 (T, U)	& R			& apply		\\
UnaryOperator<T>	& 1 (T)		& T			& apply		\\
BinaryOperator<T>	& 2 (T, T)	& T			& apply		\\
\hline																% inserts single line
\end{tabular}	
\label{table:nonlin}												% is used to refer this table in the text
\end{table}

\subsection{Implementing Supplier} 
Used to generate or supply value with taking any inputs.


\begin{mdframed}[backgroundcolor=light-gray, roundcorner=10pt,leftmargin=1, rightmargin=1, innerleftmargin=15, innertopmargin=15,innerbottommargin=15, outerlinewidth=1, linecolor=light-gray]
\begin{lstlisting}
Supplier<LocalDate> s1 = localDate::now;
Supplier<LocalDate> 21 = localDate.now();

// Static method reference
LocalDate::now; 

\end{lstlisting}
\end{mdframed} 

\subsection {Consumer and BiConsumer }
\begin{mdframed}[backgroundcolor=light-gray, roundcorner=10pt,leftmargin=1, rightmargin=1, innerleftmargin=15, innertopmargin=15,innerbottommargin=15, outerlinewidth=1, linecolor=light-gray]
\begin{lstlisting}
@Functional Interface public class Consumer<T>{
	void accept(T t);
}

Consumer<String> c1 = System.out::println;
Consumer<String> c2 = x -> System.out::println();

BiConsumer<String, Integer> c3 = map.put(k,v);
BiConsumer<String, Integer> c4 = map::put;
\end{lstlisting}
\end{mdframed} 

Consumers take a parameter and return nothing. Thus a consumer of println works. 
c1.accept("annie");

\subsection{Implementing Predicate and BiPredicate}
Predicate is often used when filter or matching. Both are common operations.

Interesting to note that when a method reference is used on something such as a particular string the following can be used:
\\
\\ \textit{Method reference}
\\ Predicate<String> p1 = String::isEmpty;
\\
\\ \textit{Without the method reference (lambda expression)}
\\ Predicate<String> p2 = x-> x.isEmpty();

\subsection{Implementing Function and BiFunction}
A Function is responsible for turning one parameter into a value of a potentially different type and returning it.
Similarly, a BiFunction is responsible for turning two turning two parameters into a value and returning it.
\begin{mdframed}[backgroundcolor=light-gray, roundcorner=10pt,leftmargin=1, rightmargin=1, innerleftmargin=15, innertopmargin=15,innerbottommargin=15, outerlinewidth=1, linecolor=light-gray]
\begin{lstlisting}
@FunctionalInterface public class Function<T,R> {
	R apply(T t);
}
@FunctionalInterface public class BiFunction<T,U,R> {
	R apply(T t, U u);
}
\end{lstlisting}
\end{mdframed} 
R is the return type from the function. \\

A another thing to remember about functions is that UnaryOperator and BinaryOperator both inherit function.
Memorize table 4.1 but it is also easy to remember that all the classes that inherit from Function take apply. 
Consumer uses accept, Predicate test, and Supplier get. \\
UnaryOperator and BinaryOperator require the type to be the same hence why R does not need to be provided. \textbf{Think types} when working with Functions.

\section{ Returning an Option }
Optional is useful for checking if a value exist. If one does not exist use Optional.empty();
If an optional does exist then to get the value use opt.get(), and to check first use opt.isPresent();
To return an Optional use Optional.of(...);

% Important note partionBy() always returns a map with two keys while 
% group by only returns one if one is needed.

\end{document}
