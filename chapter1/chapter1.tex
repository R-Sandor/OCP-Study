%file: chapter1.tex
\documentclass[12pt]{article}
\usepackage{blindtext}
\usepackage[T1]{fontenc}
\usepackage{listings} %code extracts
\usepackage{xcolor} %custom colors
\usepackage{mdframed} %nice frames
\usepackage{lmodern}
\usepackage{tikz}
\usepackage{tabularx, ragged2e, booktabs}
\usepackage{makecell}
\newcolumntype{Z}{ >{\centering\arraybackslash}X }
\definecolor{light-gray}{gray}{0.95} %the shade of grey that stack exchange uses

\def\checkmark{\tikz\fill[scale=0.4](0,.35) -- (.25,0) -- (1,.7) -- (.25,.15) -- cycle;}
\begin{document}
\title {Chapter 1 OCP Notes}
\author {Raphael J. Sandor}
\maketitle
\subsection*{Java Class Design}
\begin{itemize} 	
	\item Implement inheritance including visibility modifier and composition. \checkmark
	\item Implement polymorphism. \checkmark
	\item Override hashCode, equals, and toString methods from Object class. \checkmark
	\item Develop code that uses the static keyword on initialize blocks, variables, and classes. \checkmark
\end{itemize}

\subsection*{Advanced Java Class Design }
\begin{itemize} 	
	\item Develop code that uses abstract classes and methods.  \checkmark
	\item Develop code that uses final keyword. \checkmark
	\item Create inner classes including static inner class, local class, nested class, and anonymous inner class.  \checkmark
	\item Use enumerated types including methods and constructors in an enum type. \checkmark
	\item Develop code that declares, implements, and/or extends interface and uses the @Override annotation. \checkmark
\end{itemize}

\section*{Notes}


\begin{table}
\setlength\extrarowheight{2pt} 
\caption{Access Modifiers}	% title of Table
\centering																							% used for centering table
\begin{tabularx}{\textwidth}{*{5}{Z}}
\toprule
%\multicolumn{5}{c}{xxx xx xxxx}  \\
\thead{Can access}  
& \thead{Member \\ is private} 
& \thead{Member \\ has Default \\ (package \\ private)\\ access} 
& \thead{Member \\ is protected} 
& \thead{Member \\ is public}  \\ 
%heading
\hline																							% inserts single horizontal line
Member in the same class 						& Yes	& Yes	& Yes	& Yes		\\			% inserting body of the table 
\midrule
Member in another class in the same package		& No	& Yes	& Yes	& Yes		\\ [1ex]	% [1ex] adds vertical space
\midrule
Member in a superclass in a different package	& No 	& No	& Yes	& Yes		\\
\midrule
Method/field in a class (that is not a superclass) in a different package
												& No	& No	& No	& Yes		\\
\hline																							% inserts single line
\end{tabularx}	
\label{table:nonlin}																			% is used to refer this table in the text
\end{table}


\end{document}
