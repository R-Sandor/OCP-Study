%file: chapter1.tex
\documentclass[12pt]{article}
\usepackage{blindtext}
\usepackage[T1]{fontenc}
\usepackage{listings} %code extracts
\usepackage{xcolor} %custom colors
\usepackage{mdframed} %nice frames
\usepackage{lmodern}
\usepackage{tikz}

\definecolor{light-gray}{gray}{0.95} %the shade of grey that stack exchange uses

\def\checkmark{\tikz\fill[scale=0.4](0,.35) -- (.25,0) -- (1,.7) -- (.25,.15) -- cycle;}
\begin{document}
\title {Chapter 1 OCP Notes}
\author {Raphael J. Sandor}
\maketitle
\subsection*{Java Class Design}
\begin{itemize} 	
	\item Implement inheritance including visibility modifier and composition. \checkmark
	\item Implement polymorphism. \checkmark
	\item Override hashCode, equals, and toString methods from Object class. \checkmark
	\item Develop code that uses the static keyword on initialize blocks, variables, and classes. \checkmark
\end{itemize}

\subsection*{Advanced Java Class Design }
\begin{itemize} 	
	\item Develop code that uses abstract classes and methods.  \checkmark
	\item Develop code that uses final keyword. \checkmark
	\item Create inner classes including static inner class, local class, nested class, and annonymous inner class.  \checkmark
	\item Use enumerated types including methods and constructors in an enum type. \checkmark
	\item Develop code that declares, implements, and/or extends interface and uses the @Override annotation. \checkmark
\end{itemize}

\section*{Notes}

%
%\begin{table}[ht]
%\caption{Common Functional Interfaces}	% title of Table
%\centering											% used for centering table
%\begin{tabular}{llll}							% centered columns (4 columns)
%\hline\hline										%inserts double horizontal lines
%Functional Interfaces & \#Parameters & Return type & Single Abstract Method \\ [.5ex]
%heading
%\hline																% inserts single horizontal line
%Supplier<T>			& 0			& T			& get	  	\\			% inserting body of the table 
%Consumer<T>			& 1 (T) 	& void		& accept  	\\ [1ex]	% [1ex] adds vertical space
%BiConsumer<T>		& 2 (T, U) 	& void		& accept	\\
%Predicate<T>		& 1 (T)		& boolean	& test		\\
%BiPredicate<T, U>	& 2 (T, U)	& boolean	& test		\\
%Function<T, R>		& 1 (T)		& R			& apply		\\
%BiFunction<T, U, R> & 2 (T, U)	& R			& apply		\\
%UnaryOperator<T>	& 1 (T)		& T			& apply		\\
%BinaryOperator<T>	& 2 (T, T)	& T			& apply		\\
%\hline																% inserts single line
%\end{tabular}	
%\label{table:nonlin}												% is used to refer this table in the text
%\end{table}


\end{document}
